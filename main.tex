\documentclass[12pt]{article}
\usepackage[utf8]{inputenc}
\usepackage{latexsym}
\usepackage{amsmath}
\usepackage{amsfonts}

\title{Assignment1 - Part 1}
\author{Aswin cs19b007, Ajith cs19b014, Aryan cs19b030}
\date{\today}

\begin{document}

\maketitle

\newpage
\begin{enumerate}
    \item\begin{enumerate}
        \item abc
        \item
    \end{enumerate}
    \item 
    \item
    \item Maintain a boolean array isPresent[n] stating whether the position in the permutation is already occupied or not.
    
    We can say that $i_1$ represents the position of 1 in the permutation, since all elements to the left of $i_1$ will be greater than 1. Set isPresent[$i_1$] as true.
    Similarly, continue for the other values, such that, if isPresent[$i_k$] is true, then keep increasing the index of isPresent till an index with value false appears. This will be the position of k in the permutation.
    
    This is similar to the Open Addressing method to avoid collisions in a hash table.
    
    The best case time complexity will be of n comparisons,\textit{(i.e)} the inversion vector is given by (n-1, n-2, .....1, 0).
    
    The worst case time complexity will be $\frac{n(n-1)}{2}$ comparisons, \textit{(i.e)} O($n^2$), when the inversion matrix is given by (0,0,0....0,0)
    in which case, for the $kth$ number, we have the isPresent array (k-1) times.
    
    TBC
    \item
    \item
    \item
    \item
    \item
\end{enumerate}

\end{document}
